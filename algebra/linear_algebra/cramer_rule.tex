\documentclass[11pt]{article}
\usepackage{amsmath}
\usepackage{xcolor}
\usepackage{tcolorbox}
%\usepackage[margin=1in]{geometry}
%\linespread{1.3}


\begin{document}

\title{Regla de Cramer}
\author{}
\date{}
\maketitle
%\tableofcontents
%\pagebreak
%\vspace*{-2cm}
\section{Problema}

Dado un sistema de ecuaciones lineales con \(n\) ecuaciones y \(n\) incógnitas:
\begin{align*}
    a_{11}\cdot x_{1} + a_{12}\cdot x_{2} + \dotsb + a_{1n}\cdot x_{n} &= b_{1}\\
    a_{21}\cdot x_{1} + a_{22}\cdot x_{2} + \dotsb + a_{2n}\cdot x_{n} &= b_{2}\\
    &\;\;\vdots \\
    a_{n1}\cdot x_{1} + a_{n2}\cdot x_{2} + \dotsb + a_{nn}\cdot x_{n} &= b_{n}
\end{align*}
y la ecuación matricial asociada al sistema:
\begin{gather*}
    \bf A\cdot x = b\\
    \mathbf{A} = \begin{bmatrix}
        a_{11} & a_{12} & \dotso & a_{1n} \\
        a_{21} & b_{22} & \dotso & a_{2n} \\
        \vdots & \vdots & \ddots & \vdots \\
        a_{n1} & a_{n2} & \dotso & a_{nn} 
        \end{bmatrix} \qquad
    \mathbf{x} = \begin{bmatrix}
        x_1\\ x_2\\ \vdots \\ x_n
        \end{bmatrix} \qquad
    \mathbf{b} = \begin{bmatrix}
        b_1\\
        b_2\\
        \vdots \\
        b_n
        \end{bmatrix}
\end{gather*}
hallar el valor de la incógnita \(x_i\) (para cualquier \(i = 1,\dotsc,n\)).

\vspace{2cm}

\section{Solución}
Según la regla de Cramer, la incógnita \(x_i\) se halla directamente mediante:

\begin{equation*}
    x_i = \frac{\left| \mathbf{A_{(i)}} \right|}{\left| \mathbf{A} \right|} = \frac{\det{\mathbf{A_{(i)}}}}{\det{\mathbf{A}}}
\end{equation*}

Donde:

\begin{itemize}
    \item \(\det{\mathbf{A}}\) es el determinante de \(\mathbf{A}\)
    \item \(\det{\mathbf{A_{(i)}}}\) es el determinante de \(\mathbf{A}\), pero 
    sustituyendo la columna \(i\) por la matriz columna de términos 
    independientes, \(\mathbf{b}\):
\end{itemize}

\begin{equation*}
    \mathbf{A_{(i)}} = \begin{bmatrix}
        a_{11} & \dotso & a_{1(i-1)} & \textcolor{red}{b_{1}} & a_{1(1+i)} & \dotso & a_{1n} \\
        a_{21} & \dotso & a_{2(i-1)} & \textcolor{red}{b_{2}} & a_{2(1+i)} & \dotso & a_{2n} \\
        \vdots & \ddots & \vdots & \textcolor{red}{\vdots} & \vdots & \ddots & \vdots \\
        a_{n1} & \dotso & a_{n(i-1)} & \textcolor{red}{b_{n}} & a_{n(1+i)} & \dotso & a_{nn}
        \end{bmatrix}
\end{equation*}

\vspace*{2cm}
\section{Ejemplo}
Para el sistema:
\begin{gather*}
    \bf A\cdot x = b\\
    \\
    \mathbf{A} = \begin{bmatrix}
        1 & \hphantom{{-}}1 & \hphantom{{-}}1 \\
        1 & {-}1 & \hphantom{{-}}2 \\
        1 & {-}1 & {-}3 
        \end{bmatrix} \qquad
    \mathbf{x} = \begin{bmatrix}
        x_1\\ x_2\\ x_3
        \end{bmatrix} \qquad
    \mathbf{b} = \begin{bmatrix}
        6\\
        5\\
        {-}10
        \end{bmatrix}
\end{gather*}
halle el valor de \(x_2\).
\vspace*{1cm}
\begin{tcolorbox}
    \begin{equation*}
        x_2 = \frac{\det{\mathbf{A_{(2)}}}}%
        {\det{\mathbf{A}}} =%
        \frac{\begin{bmatrix}
            1 & \textcolor{red}{\hphantom{{-}}6} & \hphantom{{-}}1 \\
            1 & \textcolor{red}{\hphantom{{-}}5} & \hphantom{{-}}2 \\
            1 & \textcolor{red}{{-}10} & {-}3 
            \end{bmatrix}}{\begin{bmatrix}
                1 & \hphantom{{-}}1 & \hphantom{{-}}1 \\
                1 & {-}1 & \hphantom{{-}}2 \\
                1 & {-}1 & {-}3 
                \end{bmatrix}} = \frac{20}{10}=%
            2
    \end{equation*}
\end{tcolorbox}


\end{document}